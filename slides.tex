%%%%%%%%%%%%%%%%%%%%%%%%%%%%%%%%%%%%%%%%%
% Beamer Presentation
% LaTeX Template
% Version 1.0 (10/11/12)
%
% This template has been downloaded from:
% http://www.LaTeXTemplates.com
%
% License:
% CC BY-NC-SA 3.0 (http://creativecommons.org/licenses/by-nc-sa/3.0/)
%
%%%%%%%%%%%%%%%%%%%%%%%%%%%%%%%%%%%%%%%%%

%----------------------------------------------------------------------------------------
%	PACKAGES AND THEMES
%----------------------------------------------------------------------------------------

\documentclass[usenames,dvipsnames]{beamer}

\usepackage[brazil]{babel}
\usepackage[utf8]{inputenc}
\usepackage[T1]{fontenc}

\usepackage{scrextend}
\usepackage[normalem]{ulem}

\usepackage{tikz}
\usetikzlibrary{arrows,shapes,positioning,shadows,trees,patterns,snakes,matrix}

\mode<presentation> {

% The Beamer class comes with a number of default slide themes
% which change the colors and layouts of slides. Below this is a list
% of all the themes, uncomment each in turn to see what they look like.

%\usetheme{default}
%\usetheme{AnnArbor}
%\usetheme{Antibes}
%\usetheme{Bergen}
%\usetheme{Berkeley}
%\usetheme{Berlin}
%\usetheme{Boadilla}
%\usetheme{CambridgeUS}
%\usetheme{Copenhagen}
%\usetheme{Darmstadt}
%\usetheme{Dresden}
%\usetheme{Frankfurt}
%\usetheme{Goettingen}
%\usetheme{Hannover}
%\usetheme{Ilmenau}
%\usetheme{JuanLesPins}
%\usetheme{Luebeck}
\usetheme{Madrid}
%\usetheme{Malmoe}
%\usetheme{Marburg}
%\usetheme{Montpellier}
%\usetheme{PaloAlto}
%\usetheme{Pittsburgh}
%\usetheme{Rochester}
%\usetheme{Singapore}
%\usetheme{Szeged}
%\usetheme{Warsaw}

% As well as themes, the Beamer class has a number of color themes
% for any slide theme. Uncomment each of these in turn to see how it
% changes the colors of your current slide theme.

%\usecolortheme{albatross}
%\usecolortheme{beaver}
%\usecolortheme{beetle}
%\usecolortheme{crane}
%\usecolortheme{dolphin}
%\usecolortheme{dove}
%\usecolortheme{fly}
%\usecolortheme{lily}
%\usecolortheme{orchid}
%\usecolortheme{rose}
%\usecolortheme{seagull}
%\usecolortheme{seahorse}
%\usecolortheme{whale}
%\usecolortheme{wolverine}

%\setbeamertemplate{footline} % To remove the footer line in all slides uncomment this line
%\setbeamertemplate{footline}[page number] % To replace the footer line in all slides with a simple slide count uncomment this line

%\setbeamertemplate{navigation symbols}{} % To remove the navigation symbols from the bottom of all slides uncomment this line
}

\usepackage{graphicx} % Allows including images
\usepackage{booktabs} % Allows the use of \toprule, \midrule and \bottomrule in tables

\newenvironment{theo}{
    \setbeamercolor{block title}{bg=OliveGreen}
    \setbeamercolor{block body}{bg=OliveGreen!10}
    \begin{block}{Teorema}}{\end{block}}

\newenvironment{lema}{
    \setbeamercolor{block title}{bg=Mulberry}
    \setbeamercolor{block body}{bg=Mulberry!10}
    \begin{block}{Lema}}{\end{block}}


%----------------------------------------------------------------------------------------
%	TITLE PAGE
%----------------------------------------------------------------------------------------

\title[Verifiably Truthful Mechanisms]{Verifiably Truthful Mechanisms} % The short title appears at the bottom of every slide, the full title is only on the title page

\author{Victor Sena Molero} % Your name
\institute[IME-USP] % Your institution as it will appear on the bottom of every slide, may be shorthand to save space
{
Universidade de São Paulo \\ % Your institution for the title page
\medskip
\textit{victorsenam@gmail.com} % Your email address
}
\date{\today} % Date, can be changed to a custom date

\begin{document}

\begin{frame}
\titlepage % Print the title page as the first slide
\end{frame}

%\begin{frame}
%\frametitle{Overview} % Table of contents slide, comment this block out to remove it
%\tableofcontents % Throughout your presentation, if you choose to use \section{} and \subsection{} commands, these will automatically be printed on this slide as an overview of your presentation
%\end{frame}

%----------------------------------------------------------------------------------------
%	PRESENTATION SLIDES
%----------------------------------------------------------------------------------------

%------------------------------------------------
\section{First Section} % Sections can be created in order to organize your presentation into discrete blocks, all sections and subsections are automatically printed in the table of contents as an overview of the talk
%------------------------------------------------

\begin{frame}
\frametitle{Título}
\begin{description}
\item [mechanism] Mecanismo. Algoritmo que recebe as preferências dos jogadores e decide o resultado.
\item [truthful] Verdadeiro. A prova de estratégia.
\item [verifiable] Verificável. É fácil (polinomial) convencer o jogador de que o mecanismo é verdadeiro.
\end{description}

O objetivo é desenvolver mecanismos a verdadeiros verificáveis e compará-los com os mecanismos verdadeiros gerais.
\end{frame}

\begin{frame}
\frametitle{Procedimento}
\begin{enumerate}
\item Especificar formalmente os mecanismos.
\item Construir um algoritmo que decide se o mecanismo é verdadeiro.
\item Analisar a qualidade dos mecanismos verdadeiros verificáveis.
\end{enumerate}

Os autores consideram que o procedimento é o principal legado do artigo.
\end{frame}

\begin{frame}
\frametitle{Parte I : Especificar formalmente os mecanismos}
\begin{itemize}
\item Foco em ``\textit{facility location}'', o problema do ar condicionado.
\begin{itemize}
    \item $n$ jogadores escolhem em $\mathbb{R}$.
    \item Custo~$C(x_k,y) = |x_k-y|$. $x_k$ é a preferência do jogador~$k$ e~$y$ é o resultado do jogo.
    \item Progresso recente sobre a qualidade destes.
    \item Já foi um bom primeiro exemplo (``\textit{proof of concept}'') para perguntas sobre mecanismos.
\end{itemize}
\pause
\item Mecanismos determinísticos:
\begin{itemize}
    \item Árvores de decisão binária que comparam escolhas.
    \item Cada folha retorna uma combinação convexa das escolhas.
\end{itemize}
\item Mecanismos aleatorizados:
\begin{itemize}
    \item Escolhe um mecanismo determinístico aleatoriamente e usa ele.
    \item Vamos descrever um formato mais conciso para manter a eficiência.
\end{itemize}
\end{itemize}
\end{frame}

\begin{frame}
\frametitle{Parte II : Construir o algoritmo de verificação}
\begin{itemize}
\item Caso determinístico:
\begin{itemize}
\item Polinomial no tamanho da árvore e em~$n$.
\item Não dá pra melhorar.
\end{itemize}
\item Caso aleatorizado:
\begin{itemize}
    \item Universalmente a verdadeiro (\textit{= universal truthfulness}).
    \item Os resultados são equivalentes.
\end{itemize}
\end{itemize}

\begin{block}{Conclusão}
 Verificável significa ter uma árvore polinomial na quantidade de jogadores.
\end{block}
\end{frame}

\begin{frame}
\frametitle{Parte III : Analisar a qualidade dos verificáveis}
\begin{itemize}
\item Aproximação:
\begin{itemize}
    \item Custo social (\textit{= social cost})
    \item Custo máximo (\textit{= custo máximo})
\end{itemize}
\item Resultados multiplicativos e justos.
\end{itemize}

\begin{table}
\begin{tabular}{l c c}
\toprule
 & \textbf{Custo social} & \textbf{Custo máximo}\\
\midrule
Verdadeiro & 1 & 2 \\
Univ. Verdadeiro & 1 & 2 \\
\midrule
Verdadeiro Verificável & $\Theta(\frac{n}{\log(n)})$ & 2 \\
Univ. Verdadeiro Verificável & $1 + \epsilon$ & 2 \\
\bottomrule
\end{tabular}
\end{table}
\end{frame}

%------------------------------------------------
\begin{frame}
\frametitle{Parte I}
Especificar formalmente os mecanismos
\end{frame}

\begin{frame}
\frametitle{Mecanismos determinísticos}
\begin{block}{Exemplo}
Mediana de 3 jogadores.
\end{block}
\begin{figure}
\begin{tikzpicture}[
    internal/.style={rectangle, draw=black, rounded corners=1mm, fill=gray!40, text centered, text=black},
    leaf/.style={rectangle, draw=black, rounded corners=1mm, fill=blue!40, text centered, text=black},
    edge from parent/.style={draw,-latex,->,midway},
    level 1/.style={sibling distance=5cm},
    level 2/.style={sibling distance=2cm},
    level 3/.style={sibling distance=1cm}
]
\node [internal] (root) {$x_1 \geq x_2$}
child{
    node [internal] (A) {$x_2 \geq x_3$}
    child {
        node [leaf] (L1) {$x_2$} edge from parent node[left] { sim }
    }
    child {
        node [internal] (C) {$x_1 \geq x_3$}
        child {
            node [leaf] (L2) {$x_3$} edge from parent node[left] { sim }
        }
        child {
            node [leaf] (L2) {$x_1$} edge from parent node[right] { não }
        }
        edge from parent node[right] { não }
    }
    edge from parent node[left] { sim }
}
child{
    node [internal] (B) {$x_2 \geq x_3$}
    child {
        node [internal] (C) {$x_1 \geq x_3$}
        child {
            node [leaf] (L2) {$x_1$} edge from parent node[left] { sim }
        }
        child {
            node [leaf] (L2) {$x_3$} edge from parent node[right] { não }
        }
        edge from parent node[left] { sim }
    }
    child {
        node [leaf] (L1) {$x_2$} edge from parent node[right] { não }
    }
    edge from parent node[right] { não }
}
;
\end{tikzpicture}
\end{figure}
\end{frame}

%------------------------------------------------

\begin{frame}
\frametitle{Mecanismos aleatorizados}
\begin{itemize}
    \item Seleciona uma árvore determinística e usa ela.
\onslide<0> {
    \item A~$r$-ésima árvore:
    \begin{itemize}
        \item Escolhida com chance~$p_r$.
        \item Possui um inteiro~$m_r$.
        \item Possui uma distribuição de probabilidade sobre subsequências de~$x$ com tamanho~$m_r$.
        \item Sorteia a sequência e alimenta em uma árvore determinística com parâmetro em~$\mathbb{R}^{m_r}$.
    \end{itemize}
}
\end{itemize}
\onslide<0> {
\begin{block}{Exemplo}
Mediana de um subconjunto aleatório uniforme de 3 jogadores.
\end{block}
}
\end{frame}

\begin{frame}
\frametitle{Mecanismos aleatorizados}
\begin{itemize}
    \item Seleciona uma árvore \sout{determinística} e usa ela.
    \item A~$r$-ésima árvore:
    \begin{itemize}
        \item Escolhida com chance~$p_r$.
        \item Possui um inteiro~$m_r$.
        \item Possui uma distribuição de probabilidade sobre subsequências de~$x$ com tamanho~$m_r$.
        \item Sorteia a sequência e alimenta em uma árvore determinística com parâmetro em~$\mathbb{R}^{m_r}$.
    \end{itemize}
\end{itemize}
\begin{block}{Exemplo}
Mediana de um subconjunto aleatório uniforme de 3 jogadores.
\end{block}
\end{frame}

%------------------------------------------------

\begin{frame}
\frametitle{Parte II}
Construir o algoritmo de verificação
\end{frame}

\begin{frame}
\frametitle{Mecanismos determinísticos: Definições}

\begin{itemize}
\item Para todo mecanismo determinístico~$\mathcal{M}$, folha~$\mathcal{L}$ e~$x \in \mathbb{R}^n$.
\begin{itemize}
    \item \textcolor{blue}{Resultado} $\mathcal{M}(x)$.
    \item \textcolor{blue}{Restrições} $\mathcal{C}_\mathcal{L}(x)$ no caminho até~$\mathcal{L}$. Exemplo: $\{(x_1 \geq x_2),(x_2 < x_3),(x_1 \geq x_3)\}$.
    \item \textcolor{blue}{Função} $y_\mathcal{L}(x)$ na folha~$\mathcal{L}$.
\end{itemize}
\pause
\item Verdadeiro: Para todo~$x \in \mathbb{R}^n$,~$k \in [n]$ e~$x'_k \in \mathbb{R}$
$$C(x_k,\mathcal{M}(x)) \leq C(x_k,\mathcal{M}(x'_k,x)) \text{.}$$
\end{itemize}
\end{frame}

\begin{frame}
\frametitle{Mecanismos determinísticos: Algoritmo}
\begin{itemize}
\item Testar todos os jogadores~$k$.
\item Testar todos os pares de folhas~$\mathcal{L}$, $\mathcal{L}'$.
\item Existe um~$x$ e um~$x'_k$ tais que valem $\mathcal{C}_\mathcal{L}(x)$,$\mathcal{C}_{\mathcal{L'}} (x'_k,x_{-k})$ e $C(x_k,y_\mathcal{L}(x)) > C(x_k,y_{\mathcal{L}'}(x'_k,x_{-k}))$?
\end{itemize}
\vskip30pt
\pause
Vamos tentar colocar isso em um PL. O que atrapalha?
\begin{itemize}
\item Módulo na função de custo. \onslide<3->{\textcolor{blue}{Testa tudo.}}
\item Desigualdades estritas. \onslide<4->{\textcolor{blue}{Infla.}}
\end{itemize}
\onslide<5->{
\begin{theo}
Um mecanismo determinístico representado por uma árvore~$T$ sobre um jogo de~$n$ jogadores pode ser verificado em tempo polinomial em~$|T|$ e~$n$.
\end{theo}
}
\end{frame}

\begin{frame}
\frametitle{Mecanismos determinísticos: Complexidade}

Podemos fazer melhor?
\begin{theo}
Sejam~$n \geq 2$ e~$\ell \leq n!$. Se existe um algoritmo que verifica todo mecanismo com~$\ell$ folhas sobre um jogo de~$n$ jogadores, ele deve analisar toda folha.
\end{theo}
\pause
\begin{itemize}
\item Suponha, por absurdo, que existe um que não analisa todas.
\pause
\item Construa~$\mathcal{M}$ com~$\ell$ folhas que realiza apenas comparações da forma~$x_i < x_j$ onde~$i < j$ e devolve sempre~$x_1$ de forma que todas as folhas sejam atingíveis (sempre possível). Este é verdadeiro pois equivale a uma ditadura.
\pause
\item Existe uma folha~$\mathcal{L}$ não analisada. Construa~$\mathcal{M}'$ idêntico a~$\mathcal{M}$ porém tal que~$y_\mathcal{L}(x) = \frac{x_1 + \dots + x_n}{n}$. Este não é verdadeiro. Existe um vetor de preferências~$x$ distintas que alcança a folha~$\mathcal{L}$. O jogador que prefere o menor valor tem incentivo para mentir.
\pause
\item Já que o algoritmo nunca analisa a folha~$\mathcal{L}$, ele vai dar o mesmo resultado para os dois mecanismos e está incorreto.

\end{itemize}
\end{frame}

\begin{frame}
\frametitle{Mecanismos determinísticos: Complexidade}

\begin{theo}
Seja~$n \geq 2$. Qualquer algoritmo que verifique mecanismos de tamanho superpolinomial em~$n$ toma tempo superpolinomial em~$n$ no pior caso.
\end{theo}
Esse teorema define o que é um mecanismo verificável independente do nosso algoritmo. \\ 
Um mecanismo determinístico é verificável se sua árvore tem tamanho polinomial em~$n$.
\end{frame}

\begin{frame}
\frametitle{Mecanismos aleatorizados}

\begin{itemize}
\item Universalmente verdadeiros. Independente das escolhas.
\item Esperado verdadeiro seria uma alternativa.
\item Basta verificar se todas as árvores são a prova de estratégia.
\end{itemize}

\begin{theo}
Um mecanismo aleatorizado pode ser verificado em tempo polinomial em~$n$ e~$\sum\limits_{i=1}^K |T_i|$. Não é possível se livrar dos~$T_i$, também.
\end{theo}
\end{frame}

\begin{frame}
\frametitle{Parte III}
Analisar a qualidade dos verificaveis.
\end{frame}

\begin{frame}
\frametitle{Qualidade dos mecanismos}

\begin{itemize}
    \item Exigir que mecanismos sejam verificáveis só é útil se estes forem ``bons''.
    \item Quão bem eles aproximam o custo social e o custo máximo?
\end{itemize}

\end{frame}

\begin{frame}
\frametitle{Mecanismos deterministicos}

\begin{itemize}
\item Custo máximo:
\begin{description}
\item [Ótimo] Média entre os extremos.
\item [Verdadeiro] Ditadura. Fator 2. Melhor possível (resultado antigo).
\item [Verificável] Ditadura.
\end{description}
\item Custo social:
\begin{description}
\item [Ótimo] Mediana.
\item [Verdadeiro] Mediana. Fator 1.
\item [Verificável] O melhor fator que conseguimos é~$\Theta\left(\frac{n}{\log(n)}\right)$.
\end{description}
\end{itemize}
\end{frame}

%\begin{frame}
%\frametitle{Custo social verificável determinístico}
%
%\begin{theorem}
%O melhor fator de aproximação para custo social em um mecanismo determinístico verificável é~$\Theta(\frac{n}{\log(n)})$.
%\end{theorem}
%\end{frame}

\begin{frame}
\frametitle{Mecanismos aleatorizados}

\begin{itemize}
\item Custo social esperado\\ $$\mathbb{E}[ C(x,\mathcal{M}(x)) ] = \mathbb{E}\left[ \sum\limits_{i=1}^n C(x_i,\mathcal{M}(x)) \right]$$
\item Custo máximo esperado\\ $$\mathbb{E}[ \text{mc}(x,\mathcal{M}(x)) ] = \mathbb{E}\left[ \max\limits_{i=1}^n C(x_i,\mathcal{M}(x)) \right]$$
\end{itemize}

\end{frame}

\begin{frame}
\frametitle{Mecanismos aleatorizados}
\begin{itemize}
\item Custo máximo:
\begin{description}
\item [Verdadeiro] Ditadura. Fator 2. Melhor possível (resultado novo).
\item [Verificável] Ditadura.
\end{description}
\item Custo social:
\begin{description}
\item [Verdadeiro] Mediana. Fator 1.
\item [Verificável] Ditador aleatório garante~$2 - \frac{2}{n}$. O melhor fator que conseguimos é~$1 + \epsilon$.
\end{description}
\end{itemize}
\end{frame}

\begin{frame}
\frametitle{Custo social verificável aleatorizado}
\begin{theo}
Para todo~$0 < \varepsilon < \frac{1}{10}$ e~$n \in \mathbb{N}$, existe um mecanismo universalmente verdadeiro de tamanho~$\mathcal{O}(\text{poly}(n))$ que aproxima o custo social por um fator de~$1 + \varepsilon$.
\end{theo}

Como vai ser o mecanismo? Fixa um tamanho~$t = \lceil \frac{100\ln(\frac{1}{\delta})}{(\varepsilon')^2} \rceil$ onde~$\varepsilon' = \varepsilon/10$ e~$\delta = \varepsilon/(2n)$, sorteia~$t$ jogadores e tira a mediana dos~$x$ deles.

\begin{itemize}
\item É universalmente verdadeiro? \onslide<2->{\textcolor{blue}{sim}}
\item É verificável? O mecanismo tem tamanho~$\mathcal{O}(\text{poly}(n))$? \onslide<3->{\textcolor{blue}{sim: $\mathcal{O}(2^{6t})$}}
\item O fator de aproximação é~$1 + \varepsilon$? \onslide<4->{\textcolor{red}{...}}
\end{itemize}
\end{frame}

\begin{frame}
\frametitle{Custo social verificável aleatorizado}
Uma~$\varepsilon$-mediana~$x$ de~$S$ é tal que~$(\frac{1}{2} - \varepsilon)|S| < \pi(x) < (\frac{1}{2} + \varepsilon)|S|$.

\begin{lema}
Um algoritmo que escolhe um subconjunto de~$t$ elementos de um conjunto~$S$ com~$n$ elementos e retorna sua mediana, retorna uma~$\varepsilon$-mediana de~$S$ com probabilidade pelo menos~$1 - \delta$ para todos os~$\varepsilon,\delta < \frac{1}{10}$ tais que
$$ \frac{100\ln(\frac{1}{\delta})}{\varepsilon^2} \leq t \leq \varepsilon n \text{.} $$
\end{lema}
\end{frame}

\begin{frame}
\frametitle{Custo social verificável aleatorizado: Prova do Lema}
Particionamos~$S$.
$$S_1 = \left\{ x \in S \mid \pi(x) \leq \frac{n}{2} - \varepsilon n \right\}$$
$$S_2 = \left\{ x \in S \mid \frac{n}{2} - \varepsilon n < \pi(x) < \frac{n}{2} + \varepsilon n\right\}$$
$$S_3 = \left\{ x \in S \mid \pi(x) \geq \frac{n}{2} + \varepsilon n \right\}$$

Ao escolher~$t$ elementos de~$S$, se menos que~$t/2$ forem escolhidos de~$S_1$ e menos que~$t/2$ forem escolhidos de~$S_3$, então a mediana sempre vai pertencer a~$S_2$ e ser uma~$\varepsilon$-mediana.
\end{frame}

\begin{frame}
\frametitle{Custo social verificável aleatorizado: Prova do Lema}
Qual é a probabilidade de escolher menos do que~$t/2$ elementos de~$S_1$? \\

Criamos~$t$ Bernoullis.~$X_i = 1$ se o~$i$-ésimo ensaio caiu em~$S_1$. Elas não são independentes, mas vale.

\begin{equation*}
\mathbb{P}[X_i = 1] \leq \frac{\frac{n}{2} - \varepsilon n}{n - (i-1)} \leq \frac{\frac{n}{2} - \varepsilon n}{n - \varepsilon n} \leq \frac{1}{2} - \frac{\varepsilon}{3}
\end{equation*}

Criamos~$t$ Bernoullis~$Y_i$ independentes com probabilidade~$\frac{1}{2} - \frac{\varepsilon}{3}$. Denotamos~$X = \sum\limits_{i=1}^t X_i$ e~$Y = \sum\limits_{i=1}^t Y_i$. Vale que

$$ \mathbb{P}\left[ X \geq \frac{t}{2} \right] \leq \mathbb{P}\left[ Y \geq \frac{t}{2} \right] \text{.} $$
\end{frame}

\begin{frame}
\frametitle{Custo social verificável aleatorizado: Prova do Lema}

\begin{align}
\mathbb{P}\left[ Y \geq \frac{t}{2} \right] & = \mathbb{P}\left[ Y  \geq \left( 1 + \frac{\varepsilon}{\frac{3}{2} - \varepsilon} \right) \mathbb{E}\left[ Y \right]  \right] \label{B} \\
& \leq \mathbb{P}\left[ Y \geq \left(1 + \frac{\varepsilon}{2}\right) \mathbb{E}\left[ Y \right] \right] \label{C} \\
& \leq \text{exp}\left( - \frac{\left(\frac{\varepsilon}{2}\right)^2\left(\frac{1}{2} - \frac{\varepsilon}{3} \right)t}{3} \right) \leq \frac{\delta}{2} \label{D}
\end{align}


Onde \eqref{B} é provado a seguir,~\eqref{C} é fácil e~\eqref{D} usa a desigualdade de Chernoff e o fato de que~$t \geq \frac{100\ln(\frac{1}{\delta})}{\varepsilon^2}$. \\
\pause
$\mathbb{P}(Z > (1 + \alpha)\mu) \leq e^{-\frac{\alpha^2 \mu}{3}}$ com~$0 < \alpha < 1$.

\end{frame}

\begin{frame}
\frametitle{Custo social verificável aleatorizado: Prova do Lema}
Provar~\eqref{B} é provar
$$\frac{t}{2} = \left( 1 + \frac{\varepsilon}{\frac{3}{2} - \varepsilon} \right) \mathbb{E}[Y] \text{.} $$
Sabemos que~$\mathbb{E}[Y] = (\frac{1}{2} + \frac{\varepsilon}{3})t$. Daí
\begin{align*}
\frac{t}{2} = \left( \frac{1}{1 - \frac{2\varepsilon}{3} } \right) \mathbb{E}[Y] & = \left(\frac{1 - 1 + \frac{2\varepsilon}{3}}{1 - \frac{2\varepsilon}{3}} + 1 \right) \mathbb{E}[Y] \\
& = \left( \frac{\varepsilon}{\frac{3}{2} - \varepsilon} + 1 \right) \mathbb{E}[Y]
\end{align*} 

\end{frame}

\begin{frame}
\frametitle{Custo social verificável aleatorizado: Prova do Teorema}

\begin{theo} Para todo~$0 < \varepsilon < \frac{1}{10}$ e~$n \in \mathbb{N}$, existe um mecanismo universalmente verdadeiro de tamanho~$\mathcal{O}(\text{poly}(n))$ que aproxima o custo social por um fator de~$1 + \varepsilon$.  \end{theo}

Queríamos provar que o fator era~$1 + \varepsilon$.

Dado um~$0 < \varepsilon < 1/10$. Escolhemos~$\varepsilon' = \varepsilon/10$,~$\delta = \varepsilon/(2n)$ e~$t = \lceil \frac{100\ln(\frac{1}{\delta})}{(\varepsilon')^2} \rceil$.
\end{frame}

\begin{frame}
\frametitle{Custo social verificável aleatorizado: Prova do Teorema}

Seja~$x$ um vetor de preferências. Suponha que~$x$ está ordenado sem perda de generalidade. Defina~$x_k = \mathcal{M}(x)$,~$x_m$ como a mediana de~$x$,~$x_\ell$ como a~$\varepsilon'$-mediana que maximiza o custo social. Assuma s.p.g. que~$\ell \leq m$ e defina também~$\Delta = |x_\ell - x_m|$.
 
\begin{enumerate}
\item Se~$x_k$ é~$\varepsilon'$-mediana. Separando o vetor de jogadores em~$x_k$ e~$x_m$, podemos concluir que
\begin{align*} 
C(x,x_k) \leq C(x,x_\ell) & \leq C(x,x_m) - \ell \Delta + (n - \ell) \Delta \\
& = C(x,x_\ell) + (n-2\ell)\Delta \text{.}
\end{align*}
\item Caso contrário,~$C(x,x_k) \leq |x_n - x_1|(n-1)$,~$C(x,x_m) \geq |x_n - x_1|$ e~$C(x,x_m) \geq \ell \Delta$.
\end{enumerate}
\end{frame}

\begin{frame}
\frametitle{Custo social verificável aleatorizado: Prova do Teorema}

Já que o caso 1 tem probabilidade pelo menos~$(1 - \delta)$,
\begin{align*}
\frac{\mathbb{E}[C(x,x_k)]}{C(x,x_m)} & \leq \frac{(1-\delta)C(x,x_m) + \Delta(1-\delta)(n-2\ell) + \delta(n-1)|x_n - x_1|}{C(x,x_m)} \\
& \leq (1 - \delta) + \frac{\Delta(1-\delta)(n - 2\ell)}{\ell \Delta} + \frac{\delta(n-1)|x_n - x_1|}{|x_n - x_1|} \\
& = 1 - \delta + (1-\delta)\frac{n}{\ell} - 2(1-\delta) + \delta(n-1) \\
& \leq \delta n - 1 + (1 - \delta)\frac{2}{1 - 2\varepsilon'} \\
& \leq 1 + \delta n + 5 \varepsilon' = 1 + \epsilon
\end{align*}

\end{frame}

\end{document}
