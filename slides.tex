%%%%%%%%%%%%%%%%%%%%%%%%%%%%%%%%%%%%%%%%%
% Beamer Presentation
% LaTeX Template
% Version 1.0 (10/11/12)
%
% This template has been downloaded from:
% http://www.LaTeXTemplates.com
%
% License:
% CC BY-NC-SA 3.0 (http://creativecommons.org/licenses/by-nc-sa/3.0/)
%
%%%%%%%%%%%%%%%%%%%%%%%%%%%%%%%%%%%%%%%%%

%----------------------------------------------------------------------------------------
%	PACKAGES AND THEMES
%----------------------------------------------------------------------------------------

\documentclass{beamer}

\usepackage[brazil]{babel}
\usepackage[utf8]{inputenc}
\usepackage[T1]{fontenc}

\usepackage{scrextend}

\usepackage{tikz}
\usetikzlibrary{arrows,shapes,positioning,shadows,trees,patterns,snakes,matrix}

\mode<presentation> {

% The Beamer class comes with a number of default slide themes
% which change the colors and layouts of slides. Below this is a list
% of all the themes, uncomment each in turn to see what they look like.

%\usetheme{default}
%\usetheme{AnnArbor}
%\usetheme{Antibes}
%\usetheme{Bergen}
%\usetheme{Berkeley}
%\usetheme{Berlin}
%\usetheme{Boadilla}
%\usetheme{CambridgeUS}
%\usetheme{Copenhagen}
%\usetheme{Darmstadt}
%\usetheme{Dresden}
%\usetheme{Frankfurt}
%\usetheme{Goettingen}
%\usetheme{Hannover}
%\usetheme{Ilmenau}
%\usetheme{JuanLesPins}
%\usetheme{Luebeck}
\usetheme{Madrid}
%\usetheme{Malmoe}
%\usetheme{Marburg}
%\usetheme{Montpellier}
%\usetheme{PaloAlto}
%\usetheme{Pittsburgh}
%\usetheme{Rochester}
%\usetheme{Singapore}
%\usetheme{Szeged}
%\usetheme{Warsaw}

% As well as themes, the Beamer class has a number of color themes
% for any slide theme. Uncomment each of these in turn to see how it
% changes the colors of your current slide theme.

%\usecolortheme{albatross}
%\usecolortheme{beaver}
%\usecolortheme{beetle}
%\usecolortheme{crane}
%\usecolortheme{dolphin}
%\usecolortheme{dove}
%\usecolortheme{fly}
%\usecolortheme{lily}
%\usecolortheme{orchid}
%\usecolortheme{rose}
%\usecolortheme{seagull}
%\usecolortheme{seahorse}
%\usecolortheme{whale}
%\usecolortheme{wolverine}

%\setbeamertemplate{footline} % To remove the footer line in all slides uncomment this line
%\setbeamertemplate{footline}[page number] % To replace the footer line in all slides with a simple slide count uncomment this line

%\setbeamertemplate{navigation symbols}{} % To remove the navigation symbols from the bottom of all slides uncomment this line
}

\usepackage{graphicx} % Allows including images
\usepackage{booktabs} % Allows the use of \toprule, \midrule and \bottomrule in tables

%----------------------------------------------------------------------------------------
%	TITLE PAGE
%----------------------------------------------------------------------------------------

\title[Verifiably Truthful Mechanisms]{Verifiably Truthful Mechanisms} % The short title appears at the bottom of every slide, the full title is only on the title page

\author{Victor Sena Molero} % Your name
\institute[IME-USP] % Your institution as it will appear on the bottom of every slide, may be shorthand to save space
{
Universidade de São Paulo \\ % Your institution for the title page
\medskip
\textit{victorsenam@gmail.com} % Your email address
}
\date{\today} % Date, can be changed to a custom date

\begin{document}

\begin{frame}
\titlepage % Print the title page as the first slide
\end{frame}

%\begin{frame}
%\frametitle{Overview} % Table of contents slide, comment this block out to remove it
%\tableofcontents % Throughout your presentation, if you choose to use \section{} and \subsection{} commands, these will automatically be printed on this slide as an overview of your presentation
%\end{frame}

%----------------------------------------------------------------------------------------
%	PRESENTATION SLIDES
%----------------------------------------------------------------------------------------

%------------------------------------------------
\section{First Section} % Sections can be created in order to organize your presentation into discrete blocks, all sections and subsections are automatically printed in the table of contents as an overview of the talk
%------------------------------------------------

\begin{frame}
\frametitle{Título}
\begin{description}
\item [mechanisms] Mecanismo. Algoritmo que recebe as preferências dos jogadores e decide o resultado.
\item [truthful] A prova de estratégia.
\item [verifiably] Verificável. É fácil (polinomial) convencer o agente de que o mecanismo é a prova de estratégia.
\end{description}

O objetivo é desenvolver mecanismos a prova de estratégia verificáveis e compará-los com os mecanismos a prova de estratégia genéricos.
\end{frame}

\begin{frame}
\frametitle{Procedimento}
\begin{enumerate}
\item Especificar formalmente os mecanismos.
\item Construir um algoritmo que decide se o mecanismo é a prova de estratégia.
\item Analisar a qualidade dos mecanismos verificáveis.
\end{enumerate}
\end{frame}

\begin{frame}
\frametitle{Parte I : Especificar formalmente os mecanismos}
\begin{itemize}
\item Foco em ``\textit{facility location}'', o problema do ar condicionado. Cada um dos~$n$ agentes escolhe um ponto na reta real.
\begin{itemize}
    \item Progresso recente sobre a qualidade destes.
    \item Já foi um bom primeiro exemplo (\textit{proof of concept}) para perguntas sobre mecanismos.
\end{itemize}
\pause
\item Mecanismos determinísticos.
\begin{itemize}
    \item Árvores de decisão binária.
    \item Comparam escolhas.
    \item Retornam combinação convexa das escolhas.
\end{itemize}
\item Mecanismos aleatorizados.
\begin{itemize}
    \item Escolhe um mecanismo determinístico aleatoriamente e usa ele.
    \item Vamos descrever um formato mais conciso para manter a eficiência.
\end{itemize}
\end{itemize}
\end{frame}

\begin{frame}
\frametitle{Parte II : Construir o algoritmo de verificação}
\begin{itemize}
\item Custo de um jogador é a distância entre o seu ponto favorito~$x$ (informação privada) e o ponto~$y$ escolhido pelo mecanismo. Denotamos~$C(x,y) = |x-y|$.
\item Verificar determinístico é polinomial no tamanho da árvore de decisão e quantidade de jogadores. Não dá pra fazer melhor do que isso.
\item Caso aleatorizado:
\begin{itemize}
    \item Universalmente a prova de estratégia (\textit{= universal truthfulness}).
    \item O resultado se estende.
\end{itemize}
\end{itemize}
\end{frame}

\begin{frame}
\frametitle{Parte III : Analisar a qualidade dos verificáveis}
\begin{itemize}
\item Pela natureza da Parte II, o algoritmo é verificável quando a árvore é pequena. Polinomial na quantidade de jogadores.
\item No sentido de aproximação. Custo social (\textit{= social cost}) e custo máximo (\textit{= maximum cost}).
\item Bound multiplicativo e justo. Custo atingido restrito dividido pelo custo ótimo (mecanismo qualquer).
\end{itemize}

\begin{table}
\begin{tabular}{l c c}
\toprule
 & \textbf{Custo social} & \textbf{Custo máximo}\\
\midrule
Truthful & 1 & 2 \\
Univ. Truthful & 1 & \textcolor{red}{2} \\
\midrule
Ver. Truthful & $\Theta(\frac{n}{\log(n)})$ & 2 \\
Ver. Univ. Truthful & $1 + \epsilon$ & 2 \\
\bottomrule
\end{tabular}
\end{table}
\end{frame}

%------------------------------------------------
\begin{frame}
\frametitle{Parte I}
Especificar formalmente os mecanismos
\end{frame}

\begin{frame}
\frametitle{Mecanismos determinísticos}
\begin{block}{Exemplo}
Mediana de 3 jogadores.
\end{block}
\begin{figure}
\begin{tikzpicture}[
    internal/.style={rectangle, draw=black, rounded corners=1mm, fill=gray!40, text centered, text=black},
    leaf/.style={rectangle, draw=black, rounded corners=1mm, fill=blue!40, text centered, text=black},
    edge from parent/.style={draw,-latex,->,midway},
    level 1/.style={sibling distance=5cm},
    level 2/.style={sibling distance=2cm},
    level 3/.style={sibling distance=1cm}
]
\node [internal] (root) {$x_1 \geq x_2$}
child{
    node [internal] (A) {$x_2 \geq x_3$}
    child {
        node [leaf] (L1) {$x_2$} edge from parent node[left] { sim }
    }
    child {
        node [internal] (C) {$x_1 \geq x_3$}
        child {
            node [leaf] (L2) {$x_3$} edge from parent node[left] { sim }
        }
        child {
            node [leaf] (L2) {$x_1$} edge from parent node[right] { não }
        }
        edge from parent node[right] { não }
    }
    edge from parent node[left] { sim }
}
child{
    node [internal] (B) {$x_2 \geq x_3$}
    child {
        node [internal] (C) {$x_1 \geq x_3$}
        child {
            node [leaf] (L2) {$x_1$} edge from parent node[left] { sim }
        }
        child {
            node [leaf] (L2) {$x_3$} edge from parent node[right] { não }
        }
        edge from parent node[left] { sim }
    }
    child {
        node [leaf] (L1) {$x_2$} edge from parent node[right] { não }
    }
    edge from parent node[right] { não }
}
;
\end{tikzpicture}
\end{figure}
\end{frame}

%------------------------------------------------

\begin{frame}
\frametitle{Mecanismos aleatorizados}
\begin{itemize}
\item Geralmente: Seleciona uma árvore determinística aleatoriamente e usa ela. Vamos permitir algumas coisas a mais na árvore pra facilitar (e diminuir o tamanho) da descrição de alguns mecanismos.
\item O mecanismo é representado por uma árvore com um nó raiz que escolhe aleatoriamente um de seus~$K$ filhos. O~$r$-ésimo filho é a raiz de uma árvore~$T_r$ (descrita a seguir) e é tomado com chance~$p_r$. $\sum\limits_{i=1}^k p_i = 1$.
\end{itemize}
\end{frame}

\begin{frame}
\frametitle{Mecanismos aleatorizados}
Para o~$r$-ésimo filho do nó raiz:
\begin{itemize}
    \item Existe um subconjunto fixo~$N_r$ de agentes.
    \item Existe uma distribuição de probabilidades sobre sequências~$Z = \{ z_{1}, \dots, z_{m_r} \}$ de tamanho fixo~$m_r \leq n - |N_r|$ onde cada~$z_{i}$ é uma escolha de um agente distinto fora do conjunto~$N_r$.
\end{itemize}
Após escolher um filho~$r$, o mecanismo sorteia a sequência~$Z$, constrói um vetor~$x \in \mathbb{R}^{|N_r| + m_r}$ a partir do conjunto~$N_r$ e da sequência~$Z$ e aplica um mecanismo determinístico definido por~$T_r$ sobre este vetor.
\pause
\begin{block}{Exemplo}
Mediana de um subconjunto aleatório uniforme de 3 jogadores.
\end{block}
\end{frame}

%------------------------------------------------

\begin{frame}
\frametitle{Parte II}
Construir o algoritmo de verificação
\end{frame}

\begin{frame}
\frametitle{Mecanismos determinísticos: Definições}

\begin{itemize}
\item Se~$\mathcal{M}$ é um mecanismo determinístico,~$\mathcal{L}$ é uma folha deste mecanismo e~$x$ é um vetor de escolhas dos jogadores.
\begin{itemize}
    \item $\mathcal{M}(x)$ é a escolha realizada pelo mecanismo ao receber~$x$.
    \item $\mathcal{C}_\mathcal{L}(x)$ é o conjunto de cláusulas que~$x$ deve respeitar para atingir a folha~$\mathcal{L}$. No exemplo da mediana de três, este conjunto, para a segunda folha mais à esquerda, é~$\{(x_1 \geq x_2),(x_2 < x_3),(x_1 \geq x_3)\}$.
    \item $y_\mathcal{L}(x)$ é o valor da função na folha~$\mathcal{L}$ aplicada a~$x$.
\end{itemize}
\pause
\item Um mecanismo determinístico~$\mathcal{M}$ é a prova de estratégia se, para toda entrada~$x$, todo jogador~$k$ e toda escolha alternativa~$x'_k$ para~$k$, vale que
$$C(x_k,\mathcal{M}(x)) \leq C(x_k,\mathcal{M}(x'_k,x)) \text{.}$$
\end{itemize}
\end{frame}

\begin{frame}
\frametitle{Mecanismos determinísticos: Algoritmo}
Dado um mecanismo~$\mathcal{M}$, vamos testar todo os pares de folhas~$\mathcal{L}$ e~$\mathcal{L}'$ e descobrir se existe um vetor~$x$ e uma escolha alternativa~$x'_k$ tais que:
\begin{itemize}
\item $\mathcal{C}_\mathcal{L}(x)$ e~$\mathcal{C}_{\mathcal{L}'}(x'_k,x_{-k})$ são respeitados. Isto é,~$x$ atinge a folha~$\mathcal{L}$ e, se~$k$ trocar sua escolha para~$x'_k$, atinge a folha~$\mathcal{L}'$.
\item Vale que $C(x_k,y_\mathcal{L}(x)) > C(x_k,y_{\mathcal{L}'}(x'_k,x_{-k}))$.
\end{itemize}
\end{frame}


\begin{frame}
\frametitle{Mecanismos determinísticos: Algoritmo}

\begin{itemize}
\item Denotamos por~$\hat{x}$ a concatenação de~$x$ e~$x_k'$.
\item Se não houvessem restrições de desigualdade estrita, seria fácil decidir se existe~$\hat{x}$ com PL.
\item Se $\hat{x}$ respeita, então~$\alpha \hat{x}$ respeita para todo~$\alpha \in \mathbb{R}_+$.
\item Reescrevemos qualquer desigualdade da forma~$a < b$ como~$a + 1 \leq b$. Agora o PL funciona.
\end{itemize}

\end{frame}

\begin{frame}
\frametitle{Mecanismos determinísticos: Complexidade}

\begin{theorem}
Um mecanismo determinístico representado por uma árvore~$T$ sobre um jogo de~$n$ jogadores pode ser verificado em tempo polinomial em~$|T|$ e~$n$.
\end{theorem}

O algoritmo descrito usa uma força bruta nos jogadores e nos pares de folhas da árvore. Fixados estes elementos ela monta um PL com uma quantidade polinomial em~$|T|$ de restrições e resolve o mesmo.
\end{frame}

\begin{frame}
\frametitle{Mecanismos determinísticos: Complexidade}

Podemos fazer melhor?
\begin{theorem}
Sejam~$n \geq 2$ e~$\ell \leq n!$. Se existe um algoritmo que verifica todo mecanismo com~$\ell$ folhas sobre um jogo de~$n$ jogadores, ele deve analisar toda folha.
\end{theorem}
\pause
\begin{itemize}
\item Suponha, por absurdo, que existe um que não analisa todas.
\pause
\item Construa~$\mathcal{M}$ com~$\ell$ folhas que realiza apenas comparações da forma~$x_i < x_j$ onde~$i < j$ e devolve sempre~$x_1$ de forma que todas as folhas sejam atingíveis (sempre possível). Este é a prova de estratégia, pois equivale a uma ditadura.
\pause
\item Existe uma folha~$\mathcal{L}$ não analisada. Construa~$\mathcal{M}'$ idêntico a~$\mathcal{M}$ porém tal que~$y_\mathcal{L}(x) = \frac{x_1 + \dots + x_n}{n}$. Este não é a prova de estratégia. Existe um vetor de preferências (verdadeiras)~$x$ distintas que alcança a folha~$\mathcal{L}$. O jogador que prefere o menor valor tem incentivo para mentir.
\pause
\item Já que o algoritmo nunca analisa a folha~$\mathcal{L}$, ele vai dar o mesmo resultado para os dois mecanismos e está incorreto.

\end{itemize}
\end{frame}

\begin{frame}
\frametitle{Mecanismos determinísticos: Complexidade}

\begin{theorem}
Seja~$n \geq 2$. Qualquer algoritmo que verifique mecanismos de tamanho superpolinomial em~$n$ toma tempo superpolinomial em~$n$ no pior caso.
\end{theorem}
Esse teorema define o que é um mecanismo verificável independente do nosso algoritmo. \\ 
Um mecanismo determinístico é verificável se sua árvore tem tamanho polinomial em~$n$.
\end{frame}

\begin{frame}
\frametitle{Mecanismos aleatorizados}

\begin{itemize}
\item Focamos em universalmente a prova de estratégia (\textit{= universally truthful}). Deve ser a prova de estratégia independente de todas as escolhas aleatórias.
\item Poderia ser esperado a prova de estrategia, mas não sabemos verificar estes.
\item No nosso caso, basta verificar se cada uma das árvores possíveis é a prova de estratégia.
\end{itemize}

\begin{theorem}
Um mecanismo aleatorizado pode ser verificado em tempo polinomial em~$n$ e~$\sum\limits_{i=1}^K T_i$.
\end{theorem}
\end{frame}

\begin{frame}
\frametitle{Parte III}
Analisar a qualidade dos verificaveis.
\end{frame}

\begin{frame}
\frametitle{Qualidade dos mecanismos}

\begin{itemize}
    \item Exigir que mecanismos sejam verificáveis só é útil se estes forem ``bons''.
    \item Quão bem eles aproximam o custo social e o custo máximo?
\end{itemize}

\end{frame}

\begin{frame}
\frametitle{Mecanismos deterministicos}

\begin{itemize}
\item Custo máximo:
\begin{description}
\item [Caso geral] Melhor mecanismo possível escolhe a média entre as duas escolhas extremais.
\item [Verdadeiro] Qualquer ditadura é a prova de estratégia e tem fator de aproximação 2. Esse é o melhor possível.
\item [Verificável] Ditaduras são verificáveis.
\end{description}
\item Custo social:
\begin{description}
\item [Caso geral] Melhor mecanismo possível escolhe a mediana.
\item [Verdadeiro] Medianas são a prova de estratégia. Fator de aproximação 1.
\item [Verificável] Estatísticas de ordem não são verificáveis. O melhor fator que conseguimos é~$\Theta(\frac{n}{\log(n)})$.
\end{description}
\end{itemize}

\end{frame}

\begin{frame}
\frametitle{Custo social de determinísticos verificáveis}
...
\end{frame}

\begin{frame}
\frametitle{Mecanismos aleatorizados}

\begin{itemize}
\item Custo social esperado\\ $$\mathbb{E}[ \text{sc}(x,\mathcal{M}(x)) ] = \mathbb{E}\left[ \sum\limits_{i=1}^n C(x_i,\mathcal{M}(x)) \right]$$
\item Custo máximo esperado\\ $$\mathbb{E}[ \text{mc}(x,\mathcal{M}(x)) ] = \mathbb{E}\left[ \max\limits_{i=1}^n C(x,\mathcal{M}(x)) \right]$$
\end{itemize}

\end{frame}

\begin{frame}
\end{frame}

\end{document}
